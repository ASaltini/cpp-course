\documentclass[xcolor=dvipsnames,handout]{beamer}
\usetheme{Owlie}

\def\CC{{C\nolinebreak[4]\hspace{-.05em}\raisebox{.2ex}{++}}}
\title[\CC]{Programmare in \CC}
\author[A.~Saltini]{Alessandro Saltini}
\institute[LS Tassoni]{Liceo Scientifico Statale ``A.~Tassoni''}
\date{A.S.~2015/2016}
\setlength{\parindent}{0pt}

\usepackage{polyglossia}
\setmainlanguage[variant=british]{english}

\usepackage{amssymb,amsmath,amsthm}
\usepackage{booktabs,multirow}
\usepackage{graphicx}
\usepackage{enumerate}

\usepackage{caption}
\usepackage{subcaption}
\usepackage[hidelinks,pdfusetitle]{hyperref}
\usepackage[figure]{hypcap}
\usepackage{float}

\usepackage{cool}
\Style{DSymb={\mathrm{d}},IntegrateDifferentialDSymb={\mathrm{d}}}

\usepackage[math-style=ISO,bold-style=ISO,vargreek-shape=unicode]{unicode-math}
\defaultfontfeatures{Ligatures=TeX,ExternalLocation=fonts/,Extension=.otf}
\setmathfont{XITSMath}
\setmathfont[range={\mathcal,\mathbfcal},StylisticSet=1]{XITSMath}
\defaultfontfeatures{
  Ligatures=TeX,
  ExternalLocation=fonts/,
  Extension=.otf,
  UprightFont=*R,
  ItalicFont=*I,
  BoldFont=*B,
  BoldItalicFont=*BI
}
\setmainfont{TGPagella}
\setsansfont{TGAdventor}
\setmonofont{TGCursor}

\usepackage[
  list-units=single,
  range-units=single,
  multi-part-units=single,
  exponent-product=\cdot,
  per-mode=fraction,
  math-ohm=\mathrm{\Omega},
  text-ohm={\ensuremath{\mathrm{\Omega}}},
  retain-explicit-plus=true,
  binary-units=true
]{siunitx}
\DeclareSIUnit{\atp}{at.\percent}
\DeclareSIUnit{\magn}{\ensuremath{\times}}
\DeclareSIUnit{\rpm}{rpm}
\DeclareSIUnit{\nit}{nt}
\DeclareSIUnit{\talbot}{Tb}

% \usepackage{csquotes}
\usepackage[sorting=none,isbn=true,url=false,doi=false,backend=biber]{biblatex}

\usepackage{tikz}
\usetikzlibrary{
  positioning,
  shapes,
  shadows,
  arrows,
  fit,
  decorations,
  patterns,
  mindmap
}

\usepackage{readarray}
\usepackage{ifthen}
\usepackage{pgfplots}
\usepackage{chemfig}

% \usepackage[most]{tcolorbox}
\newtcolorbox{examplebox}[1][0]{
  colback=MyPaleBlue,
  colframe=MyDarkBlue,
  colbacktitle=MyLiteBlue,enhanced,
  fonttitle=\bfseries,
  attach boxed title to top center={yshift=-2mm},
  title=#1
}


\begin{document}

\begin{frame}[noframenumbering]
  \titlepage
\end{frame}

\begin{frame}{Contenuti}
  \begin{itemize}
    \item Cos'è un computer?
    \begin{itemize}
      \item logica binaria
      \item bit come unità di informazione
      \item numeri binari (ed hex?)
      \item architettura di von Neumann
      \begin{itemize}
        \item CPU (ALU/CU)
        \item memorie (primarie / secondarie)
      \end{itemize}
    \end{itemize}
    \item Linguaggi
    \begin{itemize}
      \item assembly (1-to-1 con machine code)
      \item high-level languages
      \begin{itemize}
        \item compilation process (preprocessor -- compiler -- linker)
      \end{itemize}
    \end{itemize}
  \end{itemize}
\end{frame}

\begin{frame}{L'informatica}
  \vfill
  \begin{itemize}
    \item L'informatica \alert{non} è
    \begin{itemize}
      \item saper usare un computer
      \item saper costruire/riparare un computer
      \item usare programmi scritti da altri
    \end{itemize}
    \vfill
    \item L'informatica è
    \begin{itemize}
      \item una branca della matematica
      \item lo studio dell'\alert{informazione}
      \item lo studio degli \alert{algoritmi}
      \item lo studio dei \alert{linguaggi di programmazione}
    \end{itemize}
  \end{itemize}
  \vfill
\end{frame}

\begin{frame}{L'informazione}
  \vfill
  \begin{itemize}
    \item L'informazione si misura in \alert{bit} (\alert{bi}nary digi\alert{t})
    \vfill
    \item \SI{1}{\bit} è la quantità di informazione necessaria a determinare una
    quantità che può essere \alert{0} o \alert{1}
    \vfill
    \item Il \alert{byte} è un multiplo del bit: \SI{1}{\byte} = \SI{8}{\bit}
    \vfill
    \item Due scale di multipli del byte:
    \begin{itemize}
      \item decimale: \si{\kilo\byte} (\(10^3\)), \si{\mega\byte} (\(10^6\)), \si{\giga\byte} (\(10^9\)), \si{\tera\byte} (\(10^{12}\)), \dots
      \item \alert{binaria}: \si{\kibi\byte} (\(2^{10}\)), \si{\mebi\byte} (\(2^{20}\)), \si{\gibi\byte} (\(2^{30}\)), \si{\tebi\byte} (\(2^{40}\)), \dots
    \end{itemize}
  \end{itemize}
  \vfill
\end{frame}

\begin{frame}{Algebra Booleana}
  \vfill
  \begin{itemize}
    \item L'algebra Booleana è l'algebra dei bit
    \vfill
    \item È un \alert{modello} della logica classica: 1 = vero, 0 = falso
    \vfill
    \item Insieme di base \(\mathcal{B} = \{ \; 0, 1 \; \}\)
    \vfill
    \item Tre operazioni fondamentali:
    \begin{itemize}
      \item \alert{not} (non): \(\lnot : \mathcal{B} \to \mathcal{B}\)
      \item \alert{and} (et): \(\land : \mathcal{B}^2 \to \mathcal{B}\)
      \item \alert{or} (vel): \(\lor : \mathcal{B}^2 \to \mathcal{B}\)
    \end{itemize}
  \end{itemize}
  \vfill
\end{frame}

\begin{frame}{Algebra Booleana}
  \begin{itemize}
      \item \alert{not} (non): \(\lnot\)
      \begin{itemize}
        \item \(\lnot 1 = 0\)
        \item \(\lnot 0 = 1\)
      \end{itemize}
      \vfill
      \item \alert{and} (et): \(\land\)
      \begin{itemize}
        \item \(1 \land 1 = 1\)
        \item \(1 \land 0 = 0\)
        \item \(0 \land 1 = 0\)
        \item \(0 \land 1 = 0\)
      \end{itemize}
      \vfill
      \item \alert{or} (vel): \(\lor\)
      \begin{itemize}
        \item \(1 \lor 1 = 1\)
        \item \(1 \lor 0 = 1\)
        \item \(0 \lor 1 = 1\)
        \item \(0 \lor 1 = 0\)
      \end{itemize}
  \end{itemize}
\end{frame}

\begin{frame}{Algebra Booleana}
  \vfill
  \begin{itemize}
    \item Combinando queste tre operazioni si possono ottenere tutte le altre
    operazioni possibili
    \vfill
    \item In realtà basta \alert{una} sola operazione, meno intuitiva:
    \begin{itemize}
      \item nand (\(\uparrow\))
      \item nor (\(\downarrow\))
    \end{itemize}
    \vfill
    \item Esistono circuiti \alert{elettrici} che realizzano materialmente queste
    operazioni logiche
    \begin{itemize}
      \item segnale ``alto'' = 1
      \item segnale ``basso'' = 0
    \end{itemize}
    \vfill
    \item Sono l'elemento di base dei computer
  \end{itemize}
  \vfill
\end{frame}

\begin{frame}{Rappresentazione decimale}
  \vfill
  \begin{itemize}
    \item Un numero in rappresentazione \alert{decimale} è espresso come combinazione
    di potenze di 10
    \[\num{1064} = \num{1e3} + \num{0e2} + \num{6e1} + \num{4e0}\]
    \item I coefficienti sono compresi tra 0 e 9 (\alert{minori} di 10)
    \vfill
    \item Il massimo numero con \(n\) cifre decimali è \(10^n - 1\)
    \vfill
    \item I numeri esistono indipendentemente dalla loro rappresentazione, è solo
    un modo di scriverli
  \end{itemize}
  \vfill
\end{frame}

\begin{frame}{Rappresentazione binaria}
  \vfill
  \begin{itemize}
    \item La rappresentazione \alert{binaria} utilizza le potenze di 2
    \[\num[group-digits = false]{10110} = \num[exponent-base = 2]{1e4} + \num[exponent-base = 2]{0e3}
    + \num[exponent-base = 2]{1e2} + \num[exponent-base = 2]{1e1} + \num[exponent-base = 2]{0e0}\]
    \item I coefficienti sono soltanto 0 e 1 (\alert{minori} di 2)
    \vfill
    \item Il massimo numero con \(n\) cifre binarie è \(2^n - 1\)
    \vfill
    \item Ogni cifra è rappresentabile da un \alert{bit}
    \begin{itemize}
      \item \(n\) cifre \(\Rightarrow\) \(n\) bit
    \end{itemize}
  \end{itemize}
  \vfill
\end{frame}

\begin{frame}{Rappresentazione binaria}
  \vfill
  \begin{itemize}
    \item I computer memorizzano i numeri in binario
    \vfill
    \item Le operazioni tra essi vengono svolte da appositi circuiti, basati
    sulle operazioni Booleane
    \vfill
    \item Ogni operazione richiede un certo tempo
    \vfill
    \item Limiti di memoria/tempo impediscono di operare con numeri arbitrariamente
    grandi
  \end{itemize}
  \vfill
\end{frame}

\begin{frame}{Rappresentazione binaria}
  \vfill
  \begin{itemize}
    \item I numeri negativi devono memorizzare anche il segno
    \vfill
    \item Costo di \SI{1}{\bit} aggiuntivo
    \begin{itemize}
      \item \(s = 0 \Rightarrow +\)
      \item \(s = 1 \Rightarrow -\)
    \end{itemize}
    \vfill
    \item Spesso si ricorre a rappresentazioni alternative
    \begin{itemize}
      \item rimozione di ambiguità tra \(+0\) e \(-0\)
      \item facilità di calcolo
      \item occupano comunque \SI{1}{\bit} in più
    \end{itemize}
  \end{itemize}
  \vfill
\end{frame}

\begin{frame}{Rappresentazione binaria}
  \vfill
  \begin{itemize}
    \item La parte frazionaria è problematica da rappresentare
    \[\num[group-digits = false]{1.1011} = \num[exponent-base = 2]{1e0} + \num[exponent-base = 2]{1e-1}
    + \num[exponent-base = 2]{0e-2} + \num[exponent-base = 2]{1e-3} + \num[exponent-base = 2]{1e-4}\]
    \item Non tutti i numeri con rappresentazione decimale finità hanno rappresentazione
    binaria finita
    \vfill
    \item Non possiamo memorizzare infinite cifre
    \begin{itemize}
      \item Impossibile rappresentare i numeri irrazionali
      \item Non tutti i numeri razionali sono rappresentabili
    \end{itemize}
  \end{itemize}
  \vfill
\end{frame}

\begin{frame}{Rappresentazione binaria}
  \vfill
  \begin{itemize}
    \item Richiamiamo la notazione scientifica
    \[\num{1064.15} = \num{1.06415e3}\]
    \item Generalizzabile in binario come
    \[\num[group-digits = false]{110.1011} = \num[exponent-base = 2,group-digits = false]{1.101011e2}\]
    \item La \alert{prima} cifra della rappresentazione scientifica binaria è \alert{sempre 1}, non serve memorizzarla
  \end{itemize}
  \vfill
\end{frame}

\begin{frame}{Rappresentazione binaria}
  \vfill
    \[\num[group-digits = false]{110.1011} = \num[exponent-base = 2,group-digits = false]{1.101011e2}\]
  \begin{itemize}
    \item La parte dopo la virgola è detta \alert{mantissa} o \alert{significando}, è un numero intero
    \vfill
    \item L'\alert{esponente} di 2 è un numero intero
    \vfill
    \item Un numero frazionario viene rappresentato come coppia di numeri interi
    \begin{itemize}
      \item bit del significando \(\Rightarrow\) precisione
      \item bit dell'esponente \(\Rightarrow\) range
    \end{itemize}
  \end{itemize}
  \vfill
\end{frame}

\begin{frame}{Architettura di von Neumann}
  \vfill
  \begin{center}\begin{tikzpicture}[scale=0.6, every node/.style={scale=0.6}]
    \filldraw[fill=MidnightBlue!10!white] (0,0) rectangle (5,2);
    \draw[dashed] (0,1) -- (5,1) (2.5,1) -- (2.5,2);
    \filldraw[fill=MidnightBlue!10!white] (0,-2) rectangle ++(2,1);
    \filldraw[fill=MidnightBlue!10!white] (3,-2) rectangle ++(2,1);
    \path (2.5,0.5) node {Memoria}
          (1.25,1.5) node {Dati}
          (3.75,1.5) node {Programmi}
          (1,-1.5) node {CU}
          (4,-1.5) node {ALU};
    \draw[stealth-stealth,very thick,shorten >=1pt,shorten <=1pt] (2,-1.5) -- (3,-1.5);
    \draw[stealth-stealth,very thick,shorten >=1pt,shorten <=1pt] (1,-1) -- (1,0);
    \draw[stealth-stealth,very thick,shorten >=1pt,shorten <=1pt] (4,-1) -- (4,0);
  \end{tikzpicture}\end{center}
  \vfill
  \begin{itemize}
    \item Memoria unica per dati e programmi
    \vfill
    \item CU (Control Unit): assegna e gestisce risorse
    \vfill
    \item ALU (Arithmetic Logic Unit): compie operazioni
    \vfill
    \item ALU + CU = CPU (Central Processing Unit)
  \end{itemize}
  \vfill
\end{frame}


\end{document}
