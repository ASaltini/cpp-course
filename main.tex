\documentclass[xcolor=dvipsnames,handout]{beamer}
\usetheme{Owlie}

\def\CC{{C\nolinebreak[4]\hspace{-.05em}\raisebox{.2ex}{++}}}
\title[\CC]{Programmare in \CC}
\author[A.~Saltini]{Alessandro Saltini}
\institute[LS Tassoni]{Liceo Scientifico Statale ``A.~Tassoni''}
\date{A.S.~2015/2016}
\setlength{\parindent}{0pt}

\input{headers/default}
\usepackage[
  list-units=single,
  range-units=single,
  multi-part-units=single,
  exponent-product=\cdot,
  per-mode=fraction,
  math-ohm=\mathrm{\Omega},
  text-ohm={\ensuremath{\mathrm{\Omega}}},
  retain-explicit-plus=true,
  retain-zero-exponent,
  binary-units=true
]{siunitx}
\AtBeginDocument{\sisetup{math-rm=\mathrm, text-rm=\rmfamily}}
\DeclareSIUnit{\atp}{at.\percent}
\DeclareSIUnit{\magn}{\ensuremath{\times}}
\DeclareSIUnit{\rpm}{rpm}
\DeclareSIUnit{\nit}{nt}
\DeclareSIUnit{\talbot}{Tb}

% \input{headers/bibliography}
% \usepackage{tikz}
\usetikzlibrary{
  positioning,
  shapes,
  shapes.geometric,
  shapes.symbols,
  shapes.misc,
  shadows,
  arrows,
  fit,
  decorations,
  patterns,
  mindmap,
  calc
}

\usepackage{readarray}
\usepackage{ifthen}
\usepackage{pgfplots}
\pgfplotsset{compat=newest}
\usepackage{chemfig}

\tikzset{
    begin/.style={
        draw,
        rectangle,
        rounded corners,
        very thick,
        draw=MidnightBlue!80!black,
        fill=MidnightBlue!50
    },
    input/.style={ % requires library shapes.geometric
        draw,
        trapezium,
        trapezium left angle=60,
        trapezium right angle=120,
        very thick,
        draw=BrickRed!80!black,
        fill=BrickRed!50
    },
    output/.style={ % requires library shapes.geometric
        draw,
        trapezium,
        trapezium left angle=60,
        trapezium right angle=120,
        very thick,
        draw=Fuchsia!80!black,
        fill=Fuchsia!50
    },
    operation/.style={
        draw,
        rectangle,
        very thick,
        draw=BurntOrange,
        fill=BurntOrange!50
    },
    loop/.style={ % requires library shapes.misc
        draw,
        chamfered rectangle,
        chamfered rectangle xsep=2cm,
        very thick,
        draw=ForestGreen!80!black,
        fill=ForestGreen!50
    },
    decision/.style={ % requires library shapes.geometric
        draw,
        diamond,
        aspect=#1,
        very thick,
        draw=ForestGreen!80!black,
        fill=ForestGreen!50
    },
    decision/.default=2,
    print/.style={ % requires library shapes.symbols
        draw,
        tape,
        tape bend top=none
    },
    connection/.style={
        draw,
        thick,
        circle,
        fill=MidnightBlue
    },
    process rectangle outer width/.initial=0.15cm,
    predefined process/.style={
        rectangle,
        draw,
        append after command={
        \pgfextra{
          \draw
          ($(\tikzlastnode.north west)-(0,0.5\pgflinewidth)$)--
          ($(\tikzlastnode.north west)-(\pgfkeysvalueof{/tikz/process rectangle outer width},0.5\pgflinewidth)$)--
          ($(\tikzlastnode.south west)+(-\pgfkeysvalueof{/tikz/process rectangle outer width},+0.5\pgflinewidth)$)--
          ($(\tikzlastnode.south west)+(0,0.5\pgflinewidth)$);
          \draw
          ($(\tikzlastnode.north east)-(0,0.5\pgflinewidth)$)--
          ($(\tikzlastnode.north east)+(\pgfkeysvalueof{/tikz/process rectangle outer width},-0.5\pgflinewidth)$)--
          ($(\tikzlastnode.south east)+(\pgfkeysvalueof{/tikz/process rectangle outer width},0.5\pgflinewidth)$)--
          ($(\tikzlastnode.south east)+(0,0.5\pgflinewidth)$);
        }
        },
        text width=#1,
        align=center
    },
    predefined process/.default=1.75cm,
    man op/.style={ % requires library shapes.geometric
        draw,
        trapezium,
        shape border rotate=180,
        text width=2cm,
        align=center,
    },
    extract/.style={
        draw,
        isosceles triangle,
        isosceles triangle apex angle=60,
        shape border rotate=90
    },
    merge/.style={
        draw,
        isosceles triangle,
        isosceles triangle apex angle=60,
        shape border rotate=-90
    },
    arrow/.style={
        thick,
        ->,
        >=stealth
    },
}

% \input{headers/fancy}

\begin{document}

\begin{frame}[noframenumbering]
  \titlepage
\end{frame}

\begin{frame}{Contenuti}
  \begin{itemize}
    \item Cos'è un computer?
    \begin{itemize}
      \item logica binaria
      \item bit come unità di informazione
      \item numeri binari (ed hex?)
      \item architettura di von Neumann
      \begin{itemize}
        \item CPU (ALU/CU)
        \item memorie (primarie / secondarie)
      \end{itemize}
    \end{itemize}
    \item Linguaggi
    \begin{itemize}
      \item assembly (1-to-1 con machine code)
      \item high-level languages
      \begin{itemize}
        \item compilation process (preprocessor -- compiler -- linker)
      \end{itemize}
    \end{itemize}
  \end{itemize}
\end{frame}

\begin{frame}{L'informatica}
  \vfill
  \begin{itemize}
    \item L'informatica \alert{non} è
    \begin{itemize}
      \item saper usare un computer
      \item saper costruire/riparare un computer
      \item usare programmi scritti da altri
    \end{itemize}
    \vfill
    \item L'informatica è
    \begin{itemize}
      \item una branca della matematica
      \item lo studio dell'\alert{informazione}
      \item lo studio degli \alert{algoritmi}
      \item lo studio dei \alert{linguaggi di programmazione}
    \end{itemize}
  \end{itemize}
  \vfill
\end{frame}

\begin{frame}{L'informazione}
  \vfill
  \begin{itemize}
    \item L'informazione si misura in \alert{bit} (\alert{bi}nary digi\alert{t})
    \vfill
    \item \SI{1}{\bit} è la quantità di informazione necessaria a determinare una
    quantità che può essere \alert{0} o \alert{1}
    \vfill
    \item Il \alert{byte} è un multiplo del bit: \SI{1}{\byte} = \SI{8}{\bit}
    \vfill
    \item Due scale di multipli del byte:
    \begin{itemize}
      \item decimale: \si{\kilo\byte} (\(10^3\)), \si{\mega\byte} (\(10^6\)), \si{\giga\byte} (\(10^9\)), \si{\tera\byte} (\(10^{12}\)), \dots
      \item \alert{binaria}: \si{\kibi\byte} (\(2^{10}\)), \si{\mebi\byte} (\(2^{20}\)), \si{\gibi\byte} (\(2^{30}\)), \si{\tebi\byte} (\(2^{40}\)), \dots
    \end{itemize}
  \end{itemize}
  \vfill
\end{frame}

\begin{frame}{Algebra Booleana}
  \vfill
  \begin{itemize}
    \item L'algebra Booleana è l'algebra dei bit
    \vfill
    \item È un \alert{modello} della logica classica: 1 = vero, 0 = falso
    \vfill
    \item Insieme di base \(\mathcal{B} = \{ \; 0, 1 \; \}\)
    \vfill
    \item Tre operazioni fondamentali:
    \begin{itemize}
      \item \alert{not} (non): \(\lnot : \mathcal{B} \to \mathcal{B}\)
      \item \alert{and} (et): \(\land : \mathcal{B}^2 \to \mathcal{B}\)
      \item \alert{or} (vel): \(\lor : \mathcal{B}^2 \to \mathcal{B}\)
    \end{itemize}
  \end{itemize}
  \vfill
\end{frame}

\begin{frame}{Algebra Booleana}
  \begin{itemize}
      \item \alert{not} (non): \(\lnot\)
      \begin{itemize}
        \item \(\lnot 1 = 0\)
        \item \(\lnot 0 = 1\)
      \end{itemize}
      \vfill
      \item \alert{and} (et): \(\land\)
      \begin{itemize}
        \item \(1 \land 1 = 1\)
        \item \(1 \land 0 = 0\)
        \item \(0 \land 1 = 0\)
        \item \(0 \land 1 = 0\)
      \end{itemize}
      \vfill
      \item \alert{or} (vel): \(\lor\)
      \begin{itemize}
        \item \(1 \lor 1 = 1\)
        \item \(1 \lor 0 = 1\)
        \item \(0 \lor 1 = 1\)
        \item \(0 \lor 1 = 0\)
      \end{itemize}
  \end{itemize}
\end{frame}

\begin{frame}{Algebra Booleana}
  \vfill
  \begin{itemize}
    \item Combinando queste tre operazioni si possono ottenere tutte le altre
    operazioni possibili
    \vfill
    \item In realtà basta \alert{una} sola operazione, meno intuitiva:
    \begin{itemize}
      \item nand (\(\uparrow\))
      \item nor (\(\downarrow\))
    \end{itemize}
    \vfill
    \item Esistono circuiti \alert{elettrici} che realizzano materialmente queste
    operazioni logiche
    \begin{itemize}
      \item segnale ``alto'' = 1
      \item segnale ``basso'' = 0
    \end{itemize}
    \vfill
    \item Sono l'elemento di base dei computer
  \end{itemize}
  \vfill
\end{frame}


\end{document}
