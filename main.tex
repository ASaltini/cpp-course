\documentclass[xcolor=dvipsnames,handout]{beamer}
\usetheme{Owlie}

\def\CC{{C\nolinebreak[4]\hspace{-.05em}\raisebox{.2ex}{++}}}
\title[\CC]{Programmare in \CC}
\author[A.~Saltini]{Alessandro Saltini}
\institute[LS Tassoni]{Liceo Scientifico Statale ``A.~Tassoni''}
\date{A.S.~2015/2016}
\setlength{\parindent}{0pt}

\usepackage{polyglossia}
\setmainlanguage[variant=british]{english}

\usepackage{amssymb,amsmath,amsthm}
\usepackage{booktabs,multirow}
\usepackage{graphicx}
\usepackage{enumerate}

\usepackage{caption}
\usepackage{subcaption}
\usepackage[hidelinks,pdfusetitle]{hyperref}
\usepackage[figure]{hypcap}
\usepackage{float}

\usepackage{cool}
\Style{DSymb={\mathrm{d}},IntegrateDifferentialDSymb={\mathrm{d}}}

\usepackage[math-style=ISO,bold-style=ISO,vargreek-shape=unicode]{unicode-math}
\defaultfontfeatures{Ligatures=TeX,ExternalLocation=fonts/,Extension=.otf}
\setmathfont{XITSMath}
\setmathfont[range={\mathcal,\mathbfcal},StylisticSet=1]{XITSMath}
\defaultfontfeatures{
  Ligatures=TeX,
  ExternalLocation=fonts/,
  Extension=.otf,
  UprightFont=*R,
  ItalicFont=*I,
  BoldFont=*B,
  BoldItalicFont=*BI
}
\setmainfont{TGPagella}
\setsansfont{TGAdventor}
\setmonofont{TGCursor}

\usepackage[
  list-units=single,
  range-units=single,
  multi-part-units=single,
  exponent-product=\cdot,
  per-mode=fraction,
  math-ohm=\mathrm{\Omega},
  text-ohm={\ensuremath{\mathrm{\Omega}}},
  retain-explicit-plus=true,
  binary-units=true
]{siunitx}
\DeclareSIUnit{\atp}{at.\percent}
\DeclareSIUnit{\magn}{\ensuremath{\times}}
\DeclareSIUnit{\rpm}{rpm}
\DeclareSIUnit{\nit}{nt}
\DeclareSIUnit{\talbot}{Tb}

% \usepackage{csquotes}
\usepackage[sorting=none,isbn=true,url=false,doi=false,backend=biber]{biblatex}

% \usepackage{tikz}
\usetikzlibrary{
  positioning,
  shapes,
  shadows,
  arrows,
  fit,
  decorations,
  patterns,
  mindmap
}

\usepackage{readarray}
\usepackage{ifthen}
\usepackage{pgfplots}
\usepackage{chemfig}

% \usepackage[most]{tcolorbox}
\newtcolorbox{examplebox}[1][0]{
  colback=MyPaleBlue,
  colframe=MyDarkBlue,
  colbacktitle=MyLiteBlue,enhanced,
  fonttitle=\bfseries,
  attach boxed title to top center={yshift=-2mm},
  title=#1
}


\begin{document}

\begin{frame}[noframenumbering]
  \titlepage
\end{frame}

\begin{frame}{Contenuti}
  \begin{itemize}
    \item Cos'è un computer?
    \begin{itemize}
      \item logica binaria
      \item bit come unità di informazione
      \item numeri binari (ed hex?)
      \item architettura di von Neumann
      \begin{itemize}
        \item CPU (ALU/CU)
        \item memorie (primarie / secondarie)
      \end{itemize}
    \end{itemize}
    \item Linguaggi
    \begin{itemize}
      \item assembly (1-to-1 con machine code)
      \item high-level languages
      \begin{itemize}
        \item compilation process (preprocessor -- compiler -- linker)
      \end{itemize}
    \end{itemize}
  \end{itemize}
\end{frame}

\begin{frame}{L'informatica}
  \vfill
  \begin{itemize}
    \item L'informatica \alert{non} è
    \begin{itemize}
      \item saper usare un computer
      \item saper costruire/riparare un computer
      \item usare programmi scritti da altri
    \end{itemize}
    \vfill
    \item L'informatica è
    \begin{itemize}
      \item una branca della matematica
      \item lo studio dell'\alert{informazione}
      \item lo studio degli \alert{algoritmi}
      \item lo studio dei \alert{linguaggi di programmazione}
    \end{itemize}
  \end{itemize}
  \vfill
\end{frame}

\begin{frame}{L'informazione}
  \vfill
  \begin{itemize}
    \item L'informazione si misura in \alert{bit} (\alert{bi}nary digi\alert{t})
    \vfill
    \item \SI{1}{\bit} è la quantità di informazione necessaria a determinare una
    quantità che può essere \alert{0} o \alert{1}
    \vfill
    \item Il \alert{byte} è un multiplo del bit: \SI{1}{\byte} = \SI{8}{\bit}
    \vfill
    \item Due scale di multipli del byte:
    \begin{itemize}
      \item decimale: \si{\kilo\byte} (\(10^3\)), \si{\mega\byte} (\(10^6\)), \si{\giga\byte} (\(10^9\)), \si{\tera\byte} (\(10^{12}\)), \dots
      \item \alert{binaria}: \si{\kibi\byte} (\(2^{10}\)), \si{\mebi\byte} (\(2^{20}\)), \si{\gibi\byte} (\(2^{30}\)), \si{\tebi\byte} (\(2^{40}\)), \dots
    \end{itemize}
  \end{itemize}
  \vfill
\end{frame}

\begin{frame}{Algebra Booleana}
  \vfill
  \begin{itemize}
    \item L'algebra Booleana è l'algebra dei bit
    \vfill
    \item È un \alert{modello} della logica classica: 1 = vero, 0 = falso
    \vfill
    \item Insieme di base \(\mathcal{B} = \{ \; 0, 1 \; \}\)
    \vfill
    \item Tre operazioni fondamentali:
    \begin{itemize}
      \item \alert{not} (non): \(\lnot : \mathcal{B} \to \mathcal{B}\)
      \item \alert{and} (et): \(\land : \mathcal{B}^2 \to \mathcal{B}\)
      \item \alert{or} (vel): \(\lor : \mathcal{B}^2 \to \mathcal{B}\)
    \end{itemize}
  \end{itemize}
  \vfill
\end{frame}

\begin{frame}{Algebra Booleana}
  \begin{itemize}
      \item \alert{not} (non): \(\lnot\)
      \begin{itemize}
        \item \(\lnot 1 = 0\)
        \item \(\lnot 0 = 1\)
      \end{itemize}
      \vfill
      \item \alert{and} (et): \(\land\)
      \begin{itemize}
        \item \(1 \land 1 = 1\)
        \item \(1 \land 0 = 0\)
        \item \(0 \land 1 = 0\)
        \item \(0 \land 1 = 0\)
      \end{itemize}
      \vfill
      \item \alert{or} (vel): \(\lor\)
      \begin{itemize}
        \item \(1 \lor 1 = 1\)
        \item \(1 \lor 0 = 1\)
        \item \(0 \lor 1 = 1\)
        \item \(0 \lor 1 = 0\)
      \end{itemize}
  \end{itemize}
\end{frame}

\begin{frame}{Algebra Booleana}
  \vfill
  \begin{itemize}
    \item Combinando queste tre operazioni si possono ottenere tutte le altre
    operazioni possibili
    \vfill
    \item In realtà basta \alert{una} sola operazione, meno intuitiva:
    \begin{itemize}
      \item nand (\(\uparrow\))
      \item nor (\(\downarrow\))
    \end{itemize}
    \vfill
    \item Esistono circuiti \alert{elettrici} che realizzano materialmente queste
    operazioni logiche
    \begin{itemize}
      \item segnale ``alto'' = 1
      \item segnale ``basso'' = 0
    \end{itemize}
    \vfill
    \item Sono l'elemento di base dei computer
  \end{itemize}
  \vfill
\end{frame}


\end{document}
